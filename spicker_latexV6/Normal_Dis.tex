\section{probability distributions functions}

\begin{enumerate}
    \item “d” returns the height of the probability density function
    \item “p” returns the cumulative density function
    \item “q” returns the inverse cumulative density function (quantiles)
    \item “r” returns randomly generated numbers
\end{enumerate}
dgenom
\subsection{\fbox{Normalverteilung}}
\begin{rcode}{1}
pnorm(q = , mean = , sd = )
pnorm(q = 1.96, mean = 0, sd = 1)

\end{rcode}
q = Der Wert, bis zu dem wir \(P(X \le q)\) berechnen.\\
Die Ausgabe ist die Wahrscheinlichkeit, dass \(X \le q\) gilt.\\
\textbf{Beispiel:} berechnet \(P(X \le 1.96)\) für die Standardnormalverteilung.)\\
\begin{rcode}{1}
dnorm(x = , mean = , sd = )
dnorm(x = 0, mean = 0, sd = 1)
\end{rcode}
x = Der Wert, an dem die Dichte berechnet wird.\\
\textbf{Beispiel}: gibt die Dichte der Standardnormalverteilung bei \(x = 0\) zurück.)

\bigskip

\begin{rcode}{1}
qnorm(p = , mean = , sd = )
qnorm(p = 0.975, mean = 0, sd = 1)
\end{rcode}
p = Das Quantil, also der Wahrscheinlichkeitswert (zwischen 0 und 1).\\
Die Ausgabe ist der \(x\)-Wert, sodass \(P(X \le x) = p\) gilt.\\
\textbf{Beispiel:} liefert das 97,5\%-Quantil der Standardnormalverteilung.)
\begin{rcode}{1}
rnorm(n = , mean = , sd = )
rnorm(n = 10, mean = 0, sd = 1)
\end{rcode}
n = Anzahl der zu erzeugenden Zufallswerte.\\
Die Ausgabe sind \(n\) Zufallswerte aus der Normalverteilung.\\
\textbf{Beispiel:} erzeugt 10 Zufallswerte aus einer Standardnormalverteilung.)

\columnbreak
\subsection{\fbox{Binomialverteilung}}
size = number of trials (zero or more)
\\prob = probability of success on each trial.
\begin{rcode}{1}
pbinom(q = , size = , prob = )
pbinom(q = 5, size = 10, prob = 0.3)
\end{rcode}
q = Die Anzahl der Erfolge, bis zu der \(P(X \le q)\) berechnet wird.\\
Die Ausgabe ist die Wahrscheinlichkeit, dass in \(size\) Versuchen höchstens \(q\) Erfolge erzielt werden.\\
\textbf{Beispiel:} Berechnet \(P(X \le 5)\) für eine Binomialverteilung mit 10 Versuchen und einer Erfolgswahrscheinlichkeit von 0.3.)
\begin{rcode}{1}
dbinom(x = , size = , prob = )
dbinom(x = 3, size = 10, prob = 0.3)
\end{rcode}
x = Die Anzahl der Erfolge, für die die Wahrscheinlichkeit berechnet wird.\\
\textbf{Beispiel:} Gibt die Wahrscheinlichkeit zurück, genau 3 Erfolge in 10 Versuchen zu erzielen.)
\begin{rcode}{1}
qbinom(p = , size = , prob = )
qbinom(p = 0.975, size = 10, prob = 0.3)
\end{rcode}
p = Das Quantil, also der Wahrscheinlichkeitswert (zwischen 0 und 1).\\
Die Ausgabe ist die kleinste Anzahl von Erfolgen, sodass \(P(X \le x) \ge p\) gilt.\\
\textbf{Beispiel:} Liefert das 97,5\%-Quantil der Binomialverteilung.)
\begin{rcode}{1}
rbinom(n = , size = , prob = )
rbinom(n = 10, size = 10, prob = 0.3)
\end{rcode}
n = Anzahl der zu erzeugenden Zufallszahlen.\\
Die Ausgabe sind \(n\) Zufallszahlen, die jeweils die Anzahl der Erfolge in \(size\) Versuchen darstellen.\\
\textbf{Beispiel:} Erzeugt 10 Zufallszahlen aus einer Binomialverteilung mit 10 Versuchen und einer Erfolgswahrscheinlichkeit von 0.3.)
\newpage
\subsection{\fbox{Hypergeometrische Verteilung}}
n = Nummer der Erfolge\\
M = Nummer der Misserfolge\\
k = Wie viele Versuche es gibt\\
\begin{rcode}{1}
phyper(q = , m = , n = , k = )
phyper(q = 5, m = 20, n = 30, k = 10)
\end{rcode}
q = Die Anzahl der Erfolge, bis zu der \(P(X \le q)\) berechnet wird.\\
Die Ausgabe ist die Wahrscheinlichkeit, dass bei \(k\) Ziehungen aus einer Urne mit \(m\) Erfolgen und \(n\) Misserfolgen höchstens \(q\) Erfolge erzielt werden.\\
\textbf{Beispiel:} Berechnet \(P(X \le 5)\) für eine Hypergeometrische Verteilung mit \(m = 20\), \(n = 30\) und \(k = 10\).\\

\begin{rcode}{1}
dhyper(x = , m = , n = , k = )
dhyper(x = 3, m = 20, n = 30, k = 10)
\end{rcode}
x = Die Anzahl der Erfolge, für die die Wahrscheinlichkeit berechnet wird.\\
\textbf{Beispiel:} Gibt die Wahrscheinlichkeit zurück, genau 3 Erfolge bei 10 Ziehungen zu erzielen.\\

\bigskip

\begin{rcode}{1}
qhyper(p = , m = , n = , k = )
qhyper(p = 0.975, m = 20, n = 30, k = 10)
\end{rcode}
p = Das Quantil, also der Wahrscheinlichkeitswert (zwischen 0 und 1).\\
Die Ausgabe ist die kleinste Anzahl von Erfolgen, sodass \(P(X \le x) \ge p\) gilt.\\
\textbf{Beispiel:} Liefert das 97,5\%-Quantil der Hypergeometrischen Verteilung.\\

\begin{rcode}{1}
rhyper(nn = , m = , n = , k = )
rhyper(nn = 10, m = 20, n = 30, k = 10)
\end{rcode}
nn = Anzahl der zu erzeugenden Zufallszahlen.\\
Die Ausgabe sind \(nn\) Zufallszahlen, die jeweils die Anzahl der Erfolge in \(k\) Ziehungen darstellen.\\
\textbf{Beispiel:} Erzeugt 10 Zufallszahlen aus einer Hypergeometrischen Verteilung mit \(m = 20\), \(n = 30\) und \(k = 10\).


\newpage


\section{\fbox{Expected Value und Varianz}}
\subsection{\fbox{Discrete Random Variablen}}
Erwartungswert(Mean) und Varianz einer diskreten Zufallsvariablen $X$ mit Wahrscheinlichkeitsfunktion $p(x)$
\large{
\[
E[X] = \sum_{x} x \cdot\, p\cdot(x)
\]
\[
\operatorname{Var}(X) = E\left[X^2\right] - \left(E\left[X\right]\right)^2
\]
}
\normalsize
\begin{rcode}{1}
no_car <- 0.2
one_car <- 0.7
two_cars <- 0.1
expected_value <- (0*no_car) + (1*one_car) + (2*two_cars)
var <- ((0^2*no_car) + (1^2*one_car) + (2^2*two_cars)) - expected_value^2  
\end{rcode}
Hier Berechnen wir erst Mean und dann die Var. Um zur SD zu gelangen müssen wir sqrt()\\
Um jetzt herauszufinden Wieviele Parkplätze wir bauen müssen um 99\% der Autos parken zu können. Müssen wir die Anzahl der Häuser mal dem expected Value und Var rechnen
\begin{rcode}{1}
n <- 1000
qnorm(.99, mean = n*expected_value, sd=sqrt(n*var))
\end{rcode}
\subsection{\fbox{X ist Binomialy distributed}}
\large
\[
E[X] = n \cdot p \quad, \quad \operatorname{Var}(X) =n \cdot p \cdot (1-p)
\]
\subsection{\fbox{Brauche ich hier wohl uniformly und hyper?????}}
\normalsize
\columnbreak
\section{\fbox{Central Limit Theorem}}
\subsection{\fbox{Nach Maximum Sample size Umstellen $n$}}
\normalsize
\large{\textcolor{red}{\warning} Hier sollte alpha 0.5 sein, sonst 
Brute force \textcolor{red}{\warning}}\\
-\\
\normalsize
Quantilgleichung die bei der Normalapproximation der Binominalverteilung angewendet wird:
\large
\[
k + 0.5 = n \cdot p + \operatorname{qnorm}(\alpha) \cdot \sqrt{n \cdot p \cdot (1-p)}
\]
\normalsize
Wir wissen, das wenn alpha = 0.5, ist qnorm(0.5) $=0$.\\
Damit können wir $\sqrt{n \cdot p \cdot (1-p)}$ ignorieren!\\
- Jetzt haben wir also:\\
\large
\[
k + 0.5 = n \cdot p \quad\Longrightarrow\quad n = \frac{k+0.5}{p}
\]
\normalsize
\large{\textbf{Beispiel:}}
\normalsize
Aus den Fakultäten B (25\%) und C (30\%) stammen insgesamt 55\% aller Studierenden. Bei einer zufällig gezogenen Stichprobe der Größe n ist die Anzahl X der Studierenden aus B und C binomialverteilt, also X ~ Bin(n, 0,55). Ein Raum bietet 80 Plätze, weshalb die Bedingung. Der Raum soll mit eine Chance von 50\% ausreichen\\
P(X <= 80) >= 0,5
erfüllt sein muss. Bestimme das maximale n, für das diese Anforderung gilt.\\
\[
k = 80,\quad p = 0.55,\quad \alpha = 0.5,\quad \operatorname{qnorm}(0.5) = 0
\]\[
n = \frac{80.5}{0.55} \approx 146.36.
\]

\begin{rcode}{1}
n <- 80.5 / 0.55 #146.3636
\end{rcode}
Hier ist mein Bureforce Ansatz:
\begin{rcode}{1}
new_p <- b + c
new_n <- 130:150
x <- pbinom(80, new_n, new_p)
x
new_n[max(which(x >= 0.5))] #146
\end{rcode}