- Der Index 0 z.b. $\mu_0$ bedeutet, dass es sich um einen gegebenen Wert, und nicht um einen geschätzten Wert handelt.

\begin{center}
\fbox{I) Gauß Test:}

Hauptziel: Hier wird die Hypothese über den
Mittelwert ($\mu$) getestet
\end{center}
\begin{center}
\fboxyellow{
     {\Large{Mean \textcolor{blue}{$\mu$} ist unbekannt, wir kennen SD $\sigma$}}
}
\end{center}


\large{\textbf{Gegeben muss sein:}}
\[
H_0: \mu = \textcolor{blue}{\mu_0}, \quad H_0: \mu \leq \textcolor{blue}{\mu_0}, \quad H_0: \mu \geq \textcolor{blue}{\mu_0}
\]
\[
\begin{array}{|c|l|}
\hline
\textbf{Symbol} & \textbf{Bedeutung} \\
\hline
n & \text{Stichprobengröße} \\
\sigma_0 & \text{Standardabweichung der gesamtheit} \\
\overline{X}_{(n)} & \text{Sample Mean} \\
\hline
\end{array}
\]

\normalsize
\begin{comment}
\large{\textbf{Teststatistik:}}
\[
T = \frac{\overline{X}_{(n)} - \textcolor{blue}{\mu}}{\frac{\sigma_0}{\sqrt{n}}} \sim N(0,1)
\]
\end{comment}
\large{\textbf{Decision Rule  \(R\):}}
\[
T = \frac{\overline{X} - \textcolor{blue}{\mu_0}}{\frac{\sigma_0}{\sqrt{n}}} \in R \implies \text{reject } H_0
\]
\large{\textbf{Rejection Region \(R\):}}
\[
\begin{array}{|c|c|}
\hline
H_0 & \text{rejection region } R \\ \hline
\mu = \textcolor{blue}{\mu_0} & (-\infty, -u_{1-\frac{\alpha}{2}}) \cup (u_{1-\frac{\alpha}{2}}, \infty) \\ \hline
\mu \leq \textcolor{blue}{\mu_0} & (u_{1-\alpha}, \infty) \\ \hline
\mu \geq \textcolor{blue}{\mu_0} & (-\infty, -u_{1-\alpha}) \\ \hline
\end{array}
\]
\centering\fboxyellow{$u_{1-\frac{\alpha}{2}}$ = qnorm(1 - (alpha / 2))}

\large{\textbf{Beispiel:}}
\begin{rcode}{1}
n <- 100
sd <- 0.3
sample_mean <- 10.1
alpha <- 0.1
#H0: mu = 10, H1: mu != 10
mu0 <- 10
#Rejection region
ru <- qnorm(1 - (alpha / 2))
rl <- -qnorm(1 - (alpha / 2))
#[-inf , -1.644854] or [1.644854, inf]
#teststatistic
t <- (sample_mean - mu0) /(sd / sqrt(n))
#3.333333
t > ru
#True
#we reject h0 because we are in the rejection region
p_value <- 2* pnorm(-abs(t))
#0.0008581207
p_value < alpha
#True we reject H0 

\end{rcode}
hier vielleichtg noch z test einfpgen
\columnbreak
\begin{center}
\fbox{II) t-Test:}
\end{center}
\normalsize
Hauptziel: Hier wird die Hypothese über den Mittelwert ($\mu$) getestet.
\begin{center}
\fboxyellow{
     \Large{Mean \textcolor{blue}{$\mu$} \ul{und} SD $\sigma_0$ sind unbekannt}
}

\textcolor{red}{\warning} Mean \textcolor{blue}{$\mu_0$} wird durch $H_0$ gegeben \textcolor{red}{\warning}
\end{center}
\large{\textbf{Gegeben muss sein:}}
\[
H_0: \mu = \textcolor{blue}{\mu_0}, \quad H_0: \mu \leq \textcolor{blue}{\mu_0}, \quad H_0: \mu \geq \textcolor{blue}{\mu_0}
\]
\[
\begin{array}{|c|l|}
\hline
\textbf{Symbol} & \textbf{Bedeutung} \\
\hline
n & \text{Stichprobengröße} \\
S_{(n)} & \text{Sample SD} \\
\overline{X}_{(n)} & \text{Sample Mean} \\
\hline
\end{array}
\]

\begin{comment}
\large{\textbf{Teststatistik:}}

\[
T = \frac{\overline{X}_{(n)} - \textcolor{blue}{\mu}}{\frac{s_{(n)}}{\sqrt{n}}} \sim t_{n-1},\text{with }
s^2_{(n)} = \frac{1}{n-1} \sum_{i=1}^n (X_i - \overline{X}_{(n)})^2
\]
\end{comment}
\large{\textbf{Decision Rule:}}
\[
T = \frac{\overline{X} - \textcolor{blue}{\mu_0}}{\frac{s_{(n)}}{\sqrt{n}}} \in R \implies \text{reject } H_0
\]

\large{\textbf{Rejection Region \(R\):}}

\[
\begin{array}{|c|c|}
\hline
H_0 & \text{Rejection Region } R \\ \hline
\mu = \textcolor{blue}{\mu_0} & (-\infty, -t_{n-1, 1-\frac{\alpha}{2}}) \cup (t_{n-1, 1-\frac{\alpha}{2}}, \infty) \\ \hline
\mu \leq \textcolor{blue}{\mu_0} & (t_{n-1, 1-\alpha}, \infty) \\ \hline
\mu \geq \textcolor{blue}{\mu_0} & (-\infty, -t_{n-1, 1-\alpha}) \\ \hline
\end{array}
\]
\centering\fboxyellow{$t_{n-1, 1-\frac{\alpha}{2}}$ = qt(1-alpha/2, n-1)}\\
\textbf{exact \textcolor{red}{\warning}(nur möglich wenn wir Sample habe)\textcolor{red}{\warning}:}
\begin{rcode}{1}
t.test(x = sample, mu = mu0, alternative = "two.sided", conf.level = 1-alpha)
\end{rcode}
\large{\textbf{Approx Beispiel:}}
\begin{rcode}{1}
#H0: mu >= 250, h1: < 250
n <- 82
sample_mu <- 248
sample_sd <- 5
alpha <- 0.05
mu0 <- 250
R <- -qt(1-alpha, n-1)
#[ , -1.663884]
t <- (sample_mu - mu0) / ((sample_sd) / sqrt(n))
#-3.622154
t < R
#We reject the H0
p_value <- pt(t,n - 1)
#0.0002540167
p_value < alpha
#True We reject the H0
\end{rcode}
\textbf{P-value Berechnen:}
\begin{rcode}{1}
pt(t, n-1) #H0: mu >= mu0
1-pt(t, n-1) #H0: mu <= mu0
2*pt(-abs(t), n-1) #H0: mu = mu0
\end{rcode}

\columnbreak