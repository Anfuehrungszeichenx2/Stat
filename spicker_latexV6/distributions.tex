\section{Distributions}

\subsection*{\centering\fbox{1.1)Binominale Distribution \textcolor{red}{mit} zurücklegen}}
\begin{center}
\textcolor{red}{\warning}\textcolor{red}{\warning} Mit zurücklegen \textcolor{red}{\warning}\textcolor{red}{\warning}
\end{center}
\[ 
P(X = k) \;=\; \binom{n}{k}\,p^k\,(1-p)^{\,n-k} 
\]
\large{\textbf{Beispiel:}}
\normalsize
Wir haben 7 Weiße Bälle und 3 Rote Bälle:
Wie Wie groß ist die Wahrscheinlichkeit, dass wir in $n$=5 Zügen $k$=2 rote Bälle ziehen? $p$ ist 7/10
\[
P(X = 2) = \binom{5}{2} \left(\frac{3}{10}\right)^2 \left(\frac{7}{10}\right)^3.
\]
\begin{rcode}{1}
dbinom(x = 2, size = 5, prob = 3/10)
#0.1029193
\end{rcode}
\[
    \begin{aligned}
    &x:\quad \text{Wie viele Rote Bälle wir bekommen wollen},\\
    &size:\quad \text{Wie Oft wir ziehen},\\
    &prob:\quad \text{Die prob einen Roten Ball zu ziehen}
    \end{aligned}
\]
%%%%%
\subsection*{\centering\fbox{1.2)Hypergemoetric Distribution \textcolor{red}{ohne} zurücklegen}}
\begin{center}
\textcolor{red}{\warning}\textcolor{red}{\warning} Wie Binomial, aber ohne Zurücklegen \textcolor{red}{\warning}\textcolor{red}{\warning}
\end{center}
\[
P(X = k) 
= \frac{\binom{M}{k}\,\binom{N - M}{\,n - k}}{\binom{N}{n}}
\]
\normalsize
\[
\begin{aligned}
&N:\quad \text{Gesamtanzahl aller Elemente(z.B alle Kugeln)},\\
&M:\quad \text{Anzahl der Roten Kugeln gesamt},\\
&n:\quad \text{Wie oft wir Ziehen},\\
&k:\quad \text{Wie viele Roten wir Ziehen}
\end{aligned}
\]\begin{rcode}{1}
dhyper(x = 2, m = 3,n = 7,k = 5)
#0.4166667
\end{rcode}
\normalsize
\[
\begin{aligned}
&x:\quad \text{wie viele von den gezogenen Bällen Rot sein sollen},\\
&M:\quad \text{wie viele Rote Bälle},\\
&n:\quad \text{wie viele nicht Rote},\\
&k:\quad \text{wie viele Bälle wir ziehen}
\end{aligned}
\]
\columnbreak
%%%%%
\subsection*{\centering\fbox{1.3)Multinomial Distribution \textcolor{red}{mit} zurücklegen}}
\begin{center}
\normalsize
\textcolor{red}{\warning}\textcolor{red}{\warning} Wie Binomial aber mit mehr als zwei Optionen.\textcolor{red}{\warning}\textcolor{red}{\warning}
\end{center}

\textbf{Beispiel:}
Angenommen, wir haben $n$ = 5 Versuche.

drei mögliche Ergebnisse (z.B. rot, blau, schwarz) mit 

$Rot = \frac{15}{20},\;Grün = \frac{4}{20},\;Blau = \frac{1}{20}$.
Wir fragen: Wie groß ist die Wahrscheinlichkeit, dass genau 
$Rot$=2,\; $Grün$=2,\; $Blau$=1
\text{ auftritt?}
\[
\textbf{Formel:}\quad\frac{n!}{x_1!\,x_2!\,\cdots\,x_k!}\;p_1^{x_1}\,p_2^{x_2}\,\cdots\,p_k^{x_k},
\]
\large{\[
\frac{5!}{2!\,2!\,1!}\,\Bigl(\tfrac{15}{20}\Bigr)^2\,\Bigl(\tfrac{4}{20}\Bigr)^2\,\Bigl(\tfrac{1}{20}\Bigr)^1
= 0.3375.
\]}
\normalsize
\begin{rcode}{1}
(factorial(5) / (factorial(2) * factorial(2) * factorial(1))) *
  ((15/20)^2 * (4/20)^2 * (1/20)^1) #0.03375
#Hier auch als Funktion
dmultinom(c(2, 2, 1),prob=c(15/20,4/20, 1/20))
\end{rcode}

%%%%%%%
\subsection*{\centering\fbox{1.4)Multivariate Hypergeometric Distribution}}
\begin{center}
{\centering\fbox{\textcolor{red}{OHNE} zurücklegen}}

\normalsize
\textcolor{red}{\warning}\textcolor{red}{\warning} Wie Binomial aber mit mehr als zwei Optionen. \textcolor{red}{\warning}\textcolor{red}{\warning}
\end{center}


\textbf{Beispiel:}

Angenommen, wir haben $n$ = 5 Versuche.

drei mögliche Ergebnisse (z.B. rot, blau, schwarz) mit 

$Rot = \frac{15}{20},\;Grün = \frac{4}{20},\;Blau = \frac{1}{20}$.
Wir fragen: Wie groß ist die Wahrscheinlichkeit, dass genau 
$Rot$=2,\; $Grün$=2,\; $Blau$=1
auftritt?
\[
\textbf{Formel:}\quad
P\bigl(X_1 = k_1,\,X_2 = k_2,\,\dots,\,X_r = k_r\bigr)
= \frac{\binom{K_1}{k_1}\,\binom{K_2}{k_2}\,\dots\,\binom{K_r}{k_r}}{\binom{N}{n}}
\]
\[
\frac{\binom{15}{2}\,\binom{4}{2}\,\binom{1}{1}}{\binom{20}{5}}
\approx 0.04063467.
\]
\begin{rcode}{1}
(choose(15,2) * choose(4,2) * choose(1,1)) /choose(20,5) #0.04063467
\end{rcode}
%%%%%
\columnbreak
\subsection*{\centering\fbox{1.4)Sequentielle Ziehung \textcolor{red}{mit} Zurücklegen}}
\begin{center}
\normalsize
\textcolor{red}{\warning}\textcolor{red}{\warning} mit Zurücklegen \textcolor{red}{\warning}\textcolor{red}{\warning}
\end{center}
\normalsize

Wir haben insgesamt 20 Bälle, davon sind 15 Bälle nicht rot
und 5 Bälle sind rot. Wir wollen die Wahrscheinlichkeit erst 4
nicht rote Bälle zu ziehen und dann ein roten Ball zu ziehen. - Geometrische Verteilung
\begin{mdframed}[linecolor=yellow, linewidth=2pt]
P(Keinen roten Ball) = $\frac{15}{20}$

P(Einen roten Ball) = $\frac{5}{20}$
\end{mdframed}
\large{
\[\;P(X = 5)
= \Bigl(1 - \frac{5}{20}\Bigr)^{4}
\;\cdot\;\frac{5}{20}
\]}
\normalsize
\begin{rcode}{1}
(1 - (5 / 20))^4 * 5 / 20 #0.07910156
dgeom(x = 4, prob = 5/20) #Mit Funktion
\end{rcode}
Erst nehmen wir die chance (gegenwahrscheinlichkeit) keinen roten zu ziehen hoch 4 und dann mal die chance einen roten zu ziehen

%%%%%
\subsection*{\centering\fbox{1.6)Sequentielle Ziehung \textcolor{red}{ohne} Zurücklegen}}
\begin{center}
\normalsize
\textcolor{red}{\warning}\textcolor{red}{\warning} Ohne Zurücklegen \textcolor{red}{\warning}\textcolor{red}{\warning}
\end{center}
\normalsize
Wir haben insgesamt 20 Bälle, davon sind 15 Bälle nicht rot und 5 Bälle sind rot.
Wir wollen die Wahrscheinlichkeit erst 4 nicht rote Bälle zu ziehen und dann ein roten Ball zu ziehen. Negative hypergeometrische Verteilung
\begin{tcolorbox}[colback=white,colframe=yellow,sharp corners]
P(Keinen roten Ball) = $\frac{15}{20}$

P(Einen roten Ball) = $\frac{5}{20}$

P(einen roten Ball nach 4 Zügen) = $\frac{5}{16}$

\end{tcolorbox}
\[
P\bigl(X = 5\bigr)
= \frac{\binom{15}{4}\,\binom{5}{0}}{\binom{20}{5}} \cdot \frac{5}{16}
\]
\normalsize
Wir berechnen die Wahrscheinlichkeit 4 nicht rote Bälle zu ziehen Multipliziert mit der Wahrscheinlichkeit einen roten aus den verbleibenden Bällen zu Ziehen.
\begin{rcode}{1}
((choose(15,4)*choose(5,0))/choose(20,4)) * 5/16
#0.0880418
\end{rcode}
