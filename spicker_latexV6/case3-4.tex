\begin{center}
   \fbox{III)Test für Varianz $\textcolor{blue}{\sigma_0^2}$:}
\end{center}
\normalsize

Hauptziel: Hier wird die Hypothese über die Varianz ($\textcolor{blue}{\sigma_0^2}$) getestet.
\begin{center}
\fboxyellow{
     \Large{Mean $\mu$ \ul{und} SD $\sigma$ sind unbekannt}
}

\textcolor{red}{\warning} Kein $\sigma_0$ da $\sigma$ gegeben durch $H_0$\textcolor{red}{\warning}

\textcolor{red}{\warning} Also kein Schätzwert \textcolor{red}{\warning}
\end{center}
\large{\textbf{Gegeben muss sein:}}
\[
H_0: \sigma^2 = \textcolor{blue}{\sigma_0^2}, \quad H_0: \sigma^2 \leq \textcolor{blue}{\sigma_0^2}, \quad H_0: \sigma^2 \geq \textcolor{blue}{\sigma_0^2}
\]
\[
\begin{array}{|c|l|}
\hline
\textbf{Symbol} & \textbf{Bedeutung} \\
\hline
S_{(n)}& \text{Sample SD} \\
\overline{X}_{(n)} & \text{Sample Mean} \\
\hline
\end{array}
\]
\begin{comment}
\large{\textbf{Teststatistic:}}
\[
T \;=\; \frac{(n-1)\,S_{(n)}^2}{\textcolor{blue}{\sigma^2}}
\;\;\sim\;\;\chi^2_{n-1}
\quad\text{with}\quad
S_{(n)}^2
\;=\;
\frac{1}{n-1}\sum_{i=1}^n
\bigl(X_i - \overline{X}_{(n)}\bigr)^2.
\]
\end{comment}

\large{\textbf{Decision Rule:}}
\[
T
\;=\;
\frac{(n-1)\,S_{(n)}}{\textcolor{blue}{\sigma_0^2}}
\;\in\; R
\quad\Longrightarrow\quad
\text{reject }H_0.
\]

\large{\textbf{Rejection Region \(R\):}}
\[
\begin{array}{|c|c|}
\hline
H_0 & \text{rejection region } R \\
\hline
\sigma^2 = \textcolor{blue}{\sigma_0^2}
&
(0,\;\chi^2_{n-1,\,\tfrac{\alpha}{2}})
\;\cup\;
\bigl(\chi^2_{n-1,\,1-\tfrac{\alpha}{2}},\,\infty\bigr)
\\ \hline
\sigma^2 \leq \textcolor{blue}{\sigma_0^2}
&
\bigl(\chi^2_{n-1,\,1-\alpha},\,\infty\bigr)
\\ \hline
\sigma^2 \geq \textcolor{blue}{\sigma_0^2}
&
\bigl(0,\;\chi^2_{n-1,\,\alpha}\bigr)
\\ \hline
\end{array}
\]
\large{\textbf{Beispiel:}}
\begin{rcode}{1}
#h0: sd >= 7, h1: sd <7
n <- 82
sample_mu <- 248
sample_sd <- 5
alpha <- 0.05
sd0 <- 7
#Rejection region 
R <- qchisq(alpha, n-1)
#[ , 61.26148
#Teststatistics
t <- ((n - 1) * sample_sd)/sd0
#57.85714
t < r
#We reject H0, in R area
p_value <- pchisq(t, n-1)
#0.02419782
p_value < alpha
#we reject H0
\end{rcode}
\warning Hier noch var.test 
\columnbreak
\begin{center}
   \fbox{IIII)Bernoulli Test für Probability \textcolor{blue}{$p_0$}:}
\end{center}
\normalsize

Hauptziel: Zu prüfen, ob die beobachtete Erfolgsrate $\hat{p}$ signifikant von der vorgegebenen Wahrscheinlichkeit \textcolor{blue}{$p_0$} abweicht
\begin{center}
\fboxyellow{
     \Large{Probability \textcolor{blue}{$p_0$} ist unbekannt}
}$$
\text{Number of successes: } X = \sum_{i=1}^n X_i \sim B(n, p), \quad \text{d.h. } \mathbb{E}(X) = np
$$$$
\text{Var}(X) = np(1-p).
$$
\end{center}
\large{\textbf{Gegeben muss sein:}}

\[
H_0: p = \textcolor{blue}{\textcolor{blue}{p_0}}, \quad H_0: p \leq \textcolor{blue}{p_0}, \quad H_0: p \geq \textcolor{blue}{p_0}
\]
\[
\begin{array}{|c|l|}
\hline
\textbf{Symbol} & \textbf{Bedeutung} \\
\hline
n& \text{Stichprobengröße} \\
X & \text{Number of successes} \\
\hat{p}& \frac{X}{n} \text{ Example Probability} \\
\hline
\end{array}
\]
\large{\textbf{Teststatistic}}
$$
T = \frac{\hat{p} - \textcolor{blue}{p_0}}{\sqrt{\frac{\textcolor{blue}{p_0}(1-\textcolor{blue}{p_0})}{n}}}, \quad \text{mit } \hat{p} = \frac{X}{n}.
$$
\large{\textbf{Rejection Region $R$}}
\[
\begin{array}{|c|c|}
\hline
H_0 & \text{Rejection Area } R \\ \hline
p = \textcolor{blue}{p_0} & (-\infty, -u_{1-\frac{\alpha}{2}}) \cup (u_{1-\frac{\alpha}{2}}, \infty) \\ \hline
p \leq \textcolor{blue}{p_0} & (u_{1-\alpha}, \infty) \\ \hline
p \geq \textcolor{blue}{p_0} & (-\infty, -u_{1-\alpha}) \\ \hline
\end{array}
\]
\large{\textbf{Normal Approximation:}}
\begin{rcode}{1}
#a) 80% immunity rate
#b) H0: p <= 80, H1: p > 80
p0 <- 0.8; n <- 200; x <- 172
alpha <- 0.05
phut <- x / n
#Rejection region
R <- qnorm(1 - alpha)
#r <- [0.8289439, ]
#teststatistic
t <- (phut-p0)/sqrt((p0 * (1 - p0)) / n)
#2.12132
t > R
#We reject H0
p_value <- 1 - pnorm(t)
#0.01694743
p_value < alpha
#We reject H0
\end{rcode}
\large{\textbf{Exact test:}}
\begin{rcode}{1}
#exact
binom.test(172, p = 0.8, n = n, alternative = 'greater', conf.level = 1-alpha)
#0.01793
\end{rcode}
