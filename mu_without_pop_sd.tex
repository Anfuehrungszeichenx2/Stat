\begin{multicols*}{2}

\begin{center}
     \Large{Confidence Interval for $\mu$ (Unknown Population $\sigma$)}
\end{center}
\begin{center}
     \Large{Wir suchen den $\mu$, und wissen die Standardabweichung des Samples $S_{(n)}$}
\end{center}

$$
\left[ \bar{X}_{(n)} - t_{1 - \frac{\alpha}{2}, n-1} \cdot \frac{S_{(n)}}{\sqrt{n}}, \, \bar{X}_{(n)} + t_{1 - \frac{\alpha}{2}, n-1} \cdot \frac{S_{(n)}}{\sqrt{n}} \right]
$$
Falls wir ein Sample haben, können wir den \(t\)-Test nutzen:
\begin{lstlisting}
library(TeachingDemos)
sample <- c(8, 9, 10, 13, 14, 16, 17, 20, 21)
n <- length(sample)
sample_mean <- mean(sample)
sample_sd <- sd(sample)
alpha <- 0.05
t <- qt(1 - (alpha / 2), n - 1)
L <- sample_mean - t * (sample_sd / sqrt(n))
U <- sample_mean + t * (sample_sd / sqrt(n))
t.test(x = sample, conf.level = 1 - alpha, alternative = 'two.sided')
\end{lstlisting}
\fbox{nur Upper oder Lower}

Wir müssen alpha nicht mehr teilen, da sich die Prozente auf eine Seite konzentrieren
\begin{lstlisting}
L_alleine <- sample_mean - qt(1 - alpha, n - 1) * (sample_sd / sqrt(n))
U_alleine <- sample_mean + qt(1 - alpha, n - 1) * (sample_sd / sqrt(n))
\end{lstlisting}
\begin{center}
     \Large{Umformungen}
\end{center}
\fbox{$\bar{X}_{(n)}$: Für den Sample Mean $\bar{X}_{(n)}$ Umstellen}
$$
\bar{X}_{(n)} = \frac{\text{obere Grenze} + \text{untere Grenze}}{2}
$$
\begin{lstlisting}
sample_mean_umgestellt <- (L + U) / 2
\end{lstlisting}

\hrule
\fbox{$t_{1 - \frac{\alpha}{2}, n-1}$: Für Quantile der $t$-Verteilung $t_{1 - \frac{\alpha}{2}, n-1}$}

Aus der Intervalllänge:
$$
t_{1 - \frac{\alpha}{2}, n-1} = \frac{\text{Intervalllänge} \cdot \sqrt{n}}{2 \cdot s}
$$
Aus der oberen Grenze:
$$
t_{1 - \frac{\alpha}{2}, n-1} = \frac{\text{obere Grenze} - \bar{X}_{(n)}}{\frac{S_{(n)}}{\sqrt{n}}}
$$
Aus der unteren Grenze:
$$
t_{1 - \frac{\alpha}{2}, n-1} = \frac{\bar{X}_{(n)} - \text{untere Grenze}}{\frac{S_{(n)}}{\sqrt{n}}}
$$
\begin{lstlisting}
t_umgestellt_1 <- (U - L) * sqrt(n) / (2 * sample_sd)
t_umgestellt_2 <- (U - sample_mean) / (sample_sd / sqrt(n))
t_umgestellt_3 <- (sample_mean - L) / (sample_sd / sqrt(n))
\end{lstlisting}
\columnbreak

\fbox{$S_{(n)}$: Für die Sample Standardabweichung $S_{(n)}$}

Aus der oberen Grenze:
$$
S_{(n)} = \frac{\left( \text{obere Grenze} - \bar{X}_{(n)} \right) \cdot \sqrt{n}}{t_{1 - \frac{\alpha}{2}, n-1}}
$$
Aus der unteren Grenze:
$$
S_{(n)} = \frac{\left( \bar{X}_{(n)} - \text{untere Grenze} \right) \cdot \sqrt{n}}{t_{1 - \frac{\alpha}{2}, n-1}}
$$
\begin{lstlisting}
sample_sd_umgestellt_1 <- (U - sample_mean) * (sqrt(n)) / t
sample_sd_umgestellt_2 <- (sample_mean - L) * (sqrt(n)) / t
\end{lstlisting}

\fbox{$n$: Für die Stichprobengröße $n$}

Aus der oberen Grenze:
$$
n = \left( \frac{S_{(n)} \cdot t_{1 - \frac{\alpha}{2}, n-1}}{\text{obere Grenze} - \bar{X}_{(n)}} \right)^2
$$
Aus der unteren Grenze:
$$
n = \left( \frac{S_{(n)} \cdot t_{1 - \frac{\alpha}{2}, n-1}}{\bar{X}_{(n)} - \text{untere Grenze}} \right)^2
$$
\begin{lstlisting}
n_umgestellt_1 <- round(((sample_sd * t_value) / (U - sample_mean))^2)
n_umgestellt_2 <- round(((sample_sd * t_value) / (sample_mean - L))^2)
\end{lstlisting}

\fbox{(MOE): Mit Margin of Error }
$$
MOE = t_{1 - \frac{\alpha}{2}, n-1} \cdot \frac{S_{(n)}}{\sqrt{n}}
$$
$$
n = \left( \frac{t_{1 - \frac{\alpha}{2}, n-1} \cdot S_{(n)}}{MOE} \right)^2
$$
\begin{lstlisting}
moe <- t_value * (sample_sd / sqrt(n))
n_aus_moe <- round(((t_value * sample_sd) / moe)^2)
\end{lstlisting}
\fbox{$\alpha$: Für das Signifikanzniveau $\alpha$}
$$
\alpha = 2 \cdot \left( 1 - \Phi\left(t_{1 - \frac{\alpha}{2}, n-1}\right) \right)
$$
\begin{lstlisting}
alpha_umgestellt <- 2 * (1 - pt(qt(1 - (alpha / 2), df = n - 1), df = n-1))
\end{lstlisting}

\fbox{Intervalllänge: Formel für die Intervalllänge}

Intervalllänge = U - L
$$
\text{Intervalllänge} = 2 \cdot MOE = 2 \cdot t_{1 - \frac{\alpha}{2}, n-1} \cdot \frac{S_{(n)}}{\sqrt{n}}
$$
\begin{lstlisting}
intervalllaenge <- 2 * qt(1 - (alpha / 2), df = n - 1) * (sample_sd / sqrt(n))
\end{lstlisting}

\fbox{MOE mit Intervalllänge}
$$
MOE = \frac{\text{Intervalllänge}}{2} = t_{1 - \frac{\alpha}{2}, n-1} \cdot \frac{S_{(n)}}{\sqrt{n}}
$$
\begin{lstlisting}
moe_aus_intervalllaenge <- intervalllaenge / 2
\end{lstlisting}
\end{multicols*}