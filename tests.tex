\begin{multicols*}{2}
\raggedcolumns
- Der Index 0 z.b. $\mu_0$ bedeutet, dass es sich um einen gegebenen Wert, und nicht um einen geschätzten Wert handelt.

\begin{center}
\fbox{I) Gauß Test:}

Hauptziel: Hier wird die Hypothese über den
Mittelwert ($\mu$) getestet
\end{center}
\begin{center}
\fboxyellow{
     {\Large{Mean \textcolor{blue}{$\mu$} ist unbekannt, wir kennen SD $\sigma$}}
}
\end{center}


\large{\textbf{Gegeben muss sein:}}
\[
H_0: \mu = \textcolor{blue}{\mu_0}, \quad H_0: \mu \leq \textcolor{blue}{\mu_0}, \quad H_0: \mu \geq \textcolor{blue}{\mu_0}
\]
\[
\begin{array}{|c|l|}
\hline
\textbf{Symbol} & \textbf{Bedeutung} \\
\hline
n & \text{Stichprobengröße} \\
\sigma_0 & \text{Standardabweichung der gesamtheit} \\
\overline{X}_{(n)} & \text{Sample Mean} \\
\hline
\end{array}
\]

\normalsize
\begin{comment}
\large{\textbf{Teststatistik:}}
\[
T = \frac{\overline{X}_{(n)} - \textcolor{blue}{\mu}}{\frac{\sigma_0}{\sqrt{n}}} \sim N(0,1)
\]
\end{comment}
\large{\textbf{Decision Rule  \(R\):}}
\[
T = \frac{\overline{X} - \textcolor{blue}{\mu_0}}{\frac{\sigma_0}{\sqrt{n}}} \in R \implies \text{reject } H_0
\]
\large{\textbf{Rejection Region \(R\):}}

\[
\begin{array}{|c|c|}
\hline
H_0 & \text{rejection region } R \\ \hline
\mu = \textcolor{blue}{\mu_0} & (-\infty, -u_{1-\frac{\alpha}{2}}) \cup (u_{1-\frac{\alpha}{2}}, \infty) \\ \hline
\mu \leq \textcolor{blue}{\mu_0} & (u_{1-\alpha}, \infty) \\ \hline
\mu \geq \textcolor{blue}{\mu_0} & (-\infty, -u_{1-\alpha}) \\ \hline
\end{array}
\]

\large{\textbf{Beispiel:}}
\begin{lstlisting}
n <- 100
sd <- 0.3
sample_mean <- 10.1
alpha <- 0.1
#H0: mu = 10, H1: mu != 10
mu0 <- 10
#Rejection region
ru <- qnorm(1-(alpha/2))
rl <- -qnorm(1-(alpha/2))
#[-1.644854, 1.644854]
#teststatistic
t <- (sample_mean - mu0) / (sd / sqrt(n))
t > ru
#3.333333
#we reject h0 because we are in the rejection region
p_value <- 1 - pnorm(t)
#0.0004290603
\end{lstlisting}

\columnbreak
\begin{center}
\fbox{II) t-Test:}
\end{center}
\normalsize
Hauptziel: Hier wird die Hypothese über den Mittelwert ($\mu$) getestet.
\begin{center}
\fboxyellow{
     \Large{Mean \textcolor{blue}{$\mu$} \ul{und} SD $\sigma_0$ sind unbekannt}
}

\textcolor{red}{\warning} Mean \textcolor{blue}{\mu_0} wird durch $H_0$ gegeben \textcolor{red}{\warning}
\end{center}
\large{\textbf{Gegeben muss sein:}}
\[
H_0: \mu = \textcolor{blue}{\mu_0}, \quad H_0: \mu \leq \textcolor{blue}{\mu_0}, \quad H_0: \mu \geq \textcolor{blue}{\mu_0}
\]
\[
\begin{array}{|c|l|}
\hline
\textbf{Symbol} & \textbf{Bedeutung} \\
\hline
n & \text{Stichprobengröße} \\
S_{(n)} & \text{Sample SD} \\
\overline{X}_{(n)} & \text{Sample Mean} \\
\hline
\end{array}
\]

\begin{comment}
\large{\textbf{Teststatistik:}}

\[
T = \frac{\overline{X}_{(n)} - \textcolor{blue}{\mu}}{\frac{s_{(n)}}{\sqrt{n}}} \sim t_{n-1},\text{with }
s^2_{(n)} = \frac{1}{n-1} \sum_{i=1}^n (X_i - \overline{X}_{(n)})^2
\]
\end{comment}
\large{\textbf{Decision Rule:}}
\[
T = \frac{\overline{X} - \textcolor{blue}{\mu_0}}{\frac{s_{(n)}}{\sqrt{n}}} \in R \implies \text{reject } H_0
\]

\large{\textbf{Rejection Region \(R\):}}

\[
\begin{array}{|c|c|}
\hline
H_0 & \text{Rejection Region } R \\ \hline
\mu = \textcolor{blue}{\mu_0} & (-\infty, -t_{n-1, 1-\frac{\alpha}{2}}) \cup (t_{n-1, 1-\frac{\alpha}{2}}, \infty) \\ \hline
\mu \leq \textcolor{blue}{\mu_0} & (t_{n-1, 1-\alpha}, \infty) \\ \hline
\mu \geq \textcolor{blue}{\mu_0} & (-\infty, -t_{n-1, 1-\alpha}) \\ \hline
\end{array}
\]
\large{\textbf{Beispiel:}}
\begin{lstlisting}
#H0: mu >= 250, h1: < 250
n <- 82
sample_mu <- 248
sample_sd <- 5
alpha <- 0.05
mu0 <- 250
R <- -qt(1-alpha, n-1)
#[ , -1.663884]
t <- (sample_mu - mu0) / ((sample_sd) / sqrt(n))
#-3.622154
t < r
p_value <- pt(t,n - 1)
#0.0002540167
\end{lstlisting}
\columnbreak

\begin{center}
    \large{\textbf{III)Test für Varianz $\textcolor{orange}{\sigma_0^2}$:}}
\end{center}
\normalsize

Hauptziel: Hier wird die Hypothese über die Varianz ($\textcolor{orange}{\sigma_0^2}$) getestet.
\begin{center}
\fboxyellow{
     \Large{Mean $\mu$ \ul{und} SD $\sigma$ sind unbekannt}
}

\textcolor{red}{\warning} Kein $\sigma_0$ da $\sigma$ gegeben durch $H_0$\textcolor{red}{\warning}

\textcolor{red}{\warning} Also kein Schätzwert \textcolor{red}{\warning}
\end{center}
\large{\textbf{Gegeben muss sein:}}
\[
H_0: \sigma^2 = \textcolor{orange}{\sigma_0^2}, \quad H_0: \sigma^2 \leq \textcolor{orange}{\sigma_0^2}, \quad H_0: \sigma^2 \geq \textcolor{orange}{\sigma_0^2}
\]
\[
\begin{array}{|c|l|}
\hline
\textbf{Symbol} & \textbf{Bedeutung} \\
\hline
S_{(n)}^2& \text{Sample SD} \\
\overline{X}_{(n)} & \text{Sample Mean} \\
\hline
\end{array}
\]
\begin{comment}
\large{\textbf{Teststatistic:}}
\[
T \;=\; \frac{(n-1)\,S_{(n)}^2}{\textcolor{orange}{\sigma^2}}
\;\;\sim\;\;\chi^2_{n-1}
\quad\text{with}\quad
S_{(n)}^2
\;=\;
\frac{1}{n-1}\sum_{i=1}^n
\bigl(X_i - \overline{X}_{(n)}\bigr)^2.
\]
\end{comment}

\large{\textbf{Decision Rule:}}
\[
T
\;=\;
\frac{(n-1)\,S_{(n)}^2}{\textcolor{orange}{\sigma_0^2}}
\;\in\; R
\quad\Longrightarrow\quad
\text{reject }H_0.
\]

\large{\textbf{Rejection Region \(R\):}}
\[
\begin{array}{|c|c|}
\hline
H_0 & \text{rejection region } R \\
\hline
\sigma^2 = \textcolor{orange}{\sigma_0^2}
&
(0,\;\chi^2_{n-1,\,\tfrac{\alpha}{2}})
\;\cup\;
\bigl(\chi^2_{n-1,\,1-\tfrac{\alpha}{2}},\,\infty\bigr)
\\ \hline
\sigma^2 \leq \textcolor{orange}{\sigma_0^2}
&
\bigl(\chi^2_{n-1,\,1-\alpha},\,\infty\bigr)
\\ \hline
\sigma^2 \geq \textcolor{orange}{\sigma_0^2}
&
\bigl(0,\;\chi^2_{n-1,\,\alpha}\bigr)
\\ \hline
\end{array}
\]
\large{\textbf{Beispiel:}}
\begin{lstlisting}
#h0: sd >= 7, h1: sd <7
n <- 82
sample_mu <- 248
sample_sd <- 5
alpha <- 0.05
sd0 <- 7
#Rejection region 
R <- qchisq(alpha, n-1)
#[ ,61.26148
#Teststatistics
t <- ((n - 1) * sample_sd)/sd0
#57.85714
t < r
p_value <- pchisq(t, n-1)
#0.02419782
\end{lstlisting}
\columnbreak


\begin{center}
    \large{\textbf{IIII)Bernoulli Test für Probability \textcolor{orange}{$p_0$}:}}
\end{center}
\normalsize

Hauptziel: Zu prüfen, ob die beobachtete Erfolgsrate $\hat{p}$ signifikant von der vorgegebenen Wahrscheinlichkeit \textcolor{orange}{$p_0$} abweicht
\begin{center}
\fboxyellow{
     \Large{Probability \textcolor{orange}{$p_0$} ist unbekannt}
}$$
\text{Number of successes: } X = \sum_{i=1}^n X_i \sim B(n, p), \quad \text{d.h. } \mathbb{E}(X) = np
$$$$
\text{Var}(X) = np(1-p).
$$
\end{center}
\large{\textbf{Gegeben muss sein:}}

\[
H_0: p = \textcolor{orange}{\textcolor{orange}{p_0}}, \quad H_0: p \leq \textcolor{orange}{p_0}, \quad H_0: p \geq \textcolor{orange}{p_0}
\]
\[
\begin{array}{|c|l|}
\hline
\textbf{Symbol} & \textbf{Bedeutung} \\
\hline
n& \text{Stichprobengröße} \\
X & \text{Number of successes} \\
\hat{p}& \frac{X}{n} \text{ Example Probabilitz} \\
\hline
\end{array}
\]
\large{\textbf{Teststatistic}}
$$
T = \frac{\hat{p} - \textcolor{orange}{p_0}}{\sqrt{\frac{\textcolor{orange}{p_0}(1-\textcolor{orange}{p_0})}{n}}}, \quad \text{mit } \hat{p} = \frac{X}{n}.
$$
\large{\textbf{Decision Rule}}
$$
T = \frac{\hat{p} - \textcolor{orange}{p_0}}{\sqrt{\frac{\textcolor{orange}{p_0}(1-\textcolor{orange}{p_0})}{n}}} \in R \quad \implies \quad \text{Reject } H_0.
$$
\large{\textbf{Rejection Region $R$}}
\[
\begin{array}{|c|c|}
\hline
H_0 & \text{Rejection Area } R \\ \hline
p = \textcolor{orange}{p_0} & (-\infty, -u_{1-\frac{\alpha}{2}}) \cup (u_{1-\frac{\alpha}{2}}, \infty) \\ \hline
p \leq \textcolor{orange}{p_0} & (u_{1-\alpha}, \infty) \\ \hline
p \geq \textcolor{orange}{p_0} & (-\infty, -u_{1-\alpha}) \\ \hline
\end{array}
\]
\large{\textbf{Normal Approximation:}}
\begin{lstlisting}
#a) 80% immunity rate
#b) H0: p <= 80, H1: p > 80
p0 <- 0.8; n <- 200; x <- 172
alpha <- 0.05
phut <- x / n
#Rejection region
R <- pnorm(1 - alpha)
#r <- [0.8289439, ]
#teststatistic
t <- (phut-p0)/sqrt((p0 * (1 - p0)) / n)
#2.12132
t > R
p_value <- 1 - pnorm(t)
#0.01694743
\end{lstlisting}
\large{\textbf{Exact test:}}
\begin{lstlisting}
#exact
binom.test(172, p = 0.8, n = n, alternative = 'greater', conf.level = 1-alpha)
#0.01793
\end{lstlisting}


\end{multicols*}





