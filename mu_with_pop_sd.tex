\begin{multicols*}{2}

\begin{center}
     \Large{Confidence Interval for $\mu$ (Known Population $\sigma$)}
\end{center}
\begin{center}
     \Large{Wir suchen den $\mu$, und wissen die Standardabweichung der Population $\sigma$}
\end{center}

$$
\left[ \bar{X}_{(n)} - u_{1 - \frac{\alpha}{2}} \cdot \frac{\sigma}{\sqrt{n}}, \, \bar{X}_{(n)} + u_{1 - \frac{\alpha}{2}} \cdot \frac{\sigma}{\sqrt{n}} \right]
$$
Falls wir ein Sample haben, können wir den z Test nutzen. Wenn nicht, müssen wir die Formel so schreiben
\begin{lstlisting}
library(TeachingDemos)
sample <- c(8, 9, 10, 13, 14, 16, 17, 20, 21)
sample_mean <- mean(sample)
pop_sd <- 2.8
alpha <- 0.05
q <- qnorm(1 - (alpha / 2))
n <- length(sample)
L <- sample_mean - q * (pop_sd / sqrt(n))
U <- sample_mean + q * (pop_sd / sqrt(n))
z.test(x = sample, stdev = pop_sd, alternative = "two.sided",
       conf.level = 1-alpha)$conf.int
\end{lstlisting}
\fbox{nur Upper oder Lower}

Wir müssen alpha nicht mehr teilen, da sich die Prozente auf eine Seite konzentrieren
\begin{lstlisting}
L_alleine <- sample_mean - qnorm(1 - alpha) * (sample_sd/sqrt(n))
U_alleine <- sample_mean + qnorm(1 - alpha) * (sample_sd/sqrt(n))
\end{lstlisting}
\begin{center}
     \Large{Umformungen}
\end{center}
\fbox{$\bar{X}_{(n)}$: Für den Sample Mean $\bar{X}_{(n)}$ Umstellen}
$$
\bar{X}_{(n)} = \frac{\text{obere Grenze} + \text{untere Grenze}}{2}
$$
\begin{lstlisting}
sample_mean_umgestellt <- (L + U) / 2
\end{lstlisting}

\hrule
\fbox{$u_{1 - \frac{\alpha}{2}} \cdot \frac{\sigma}{\sqrt{n}}$: \small{Für Quantile der Normalverteilung Umformen}}
Aus der Intervalllänge:
$$u_{1 - \frac{\alpha}{2}} = \frac{\text{Intervalllänge} \cdot \sqrt{n}}{2 \cdot \sigma}$$
 Aus der oberen Grenze:
$$
u_{1 - \frac{\alpha}{2}} = \frac{\text{obere Grenze} - \bar{X}_{(n)}}{\frac{\sigma}{\sqrt{n}}}
$$
Aus der unteren Grenze:
$$
u_{1 - \frac{\alpha}{2}} = \frac{\bar{X}_{(n)} - \text{untere Grenze}}{\frac{\sigma}{\sqrt{n}}}
$$
\begin{lstlisting}
z <- (leange * sqrt(n)) / 2 * sampel_sd
q_umgestellt <-(U + sample_mean) / (pop_sd / sqrt(n))
q_umgestellt <-(sample_mean - L) / (pop_sd / sqrt(n))
\end{lstlisting}
\columnbreak
\fbox{$\sigma$: Für die Standardabweichung $\sigma$}

Aus der oberen Grenze:
$$
\sigma = \frac{\left( \text{obere Grenze} - \bar{X}_{(n)} \right) \cdot \sqrt{n}}{u_{1 - \frac{\alpha}{2}}}
$$
Aus der unteren Grenze:
$$
\sigma = \frac{\left( \bar{X}_{(n)} - \text{untere Grenze} \right) \cdot \sqrt{n}}{u_{1 - \frac{\alpha}{2}}}
$$
\begin{lstlisting}
pop_sd_umgestellt <- ((U - sample_mean) * sqrt(n)) / qnorm(1 - (alpha / 2))
pop_sd_umgestellt <- ((sample_mean - L) * sqrt(n)) / qnorm(1 - (alpha / 2))
\end{lstlisting}

\fbox{$n$: Für die Stichprobengröße $n$}

Aus der oberen Grenze:
$$
n = \left( \frac{\sigma \cdot u_{1 - \frac{\alpha}{2}}}{\text{obere Grenze} - \bar{X}_{(n)}} \right)^2
$$
Aus der unteren Grenze:
$$
n = \left( \frac{\sigma \cdot u_{1 - \frac{\alpha}{2}}}{\bar{X}_{(n)} - \text{untere Grenze}} \right)^2
$$
\begin{lstlisting}
n_umgestellt <- round(((pop_sd * q) / (U - sample_mean))^2)
n_umgestellt <- round(((pop_sd * q) / (sample_mean - L))^2)
\end{lstlisting}
\fbox{(MOE): Mit Margin of Error }
$$
MOE = u_{1 - \frac{\alpha}{2}} \cdot \frac{\sigma}{\sqrt{n}}
$$
$$
n = \left( \frac{u_{1 - \frac{\alpha}{2}} \cdot \sigma}{MOE} \right)^2
$$
\begin{lstlisting}
moe <- qnorm(1 - alpha / 2) * (pop_sd / sqrt(n))
n_aus_moe <- round(((q * pop_sd) / moe)^2)
\end{lstlisting}
\fbox{$\alpha$: Für das Signifikanzniveau $\alpha$}
$$
\alpha = 2 \cdot \left( 1 - \Phi(u_{1 - \frac{\alpha}{2}}) \right)
$$
\begin{lstlisting}
alpha_umgestellt <- 2 * (1 - pnorm(qnorm(1 - (alpha / 2))))
\end{lstlisting}
\fbox{Intervalllänge: Formel für die Intervalllänge}
MOE = U - L
$$
\text{Intervalllänge} = 2 \cdot E = 2 \cdot u_{1 - \frac{\alpha}{2}} \cdot \frac{\sigma}{\sqrt{n}}
$$
\begin{lstlisting}
intervallleange <- 2 * qnorm(1 - (alpha / 2)) * (sample_sd / sqrt(n))
\end{lstlisting}
\fbox{MOE mit Intervalllänge}

$$
E = u_{1 - \frac{\alpha}{2}} \cdot \frac{\sigma}{\sqrt{n}}
$$

\end{multicols*}