\section{Discrete Random Variables}
Bei diskreten Zufallsvariablen ordnen wir jedem möglichen Wert \(X\) – zum Beispiel den Würfelaugen \(1,2,3,4,5,6\) – eine Wahrscheinlichkeit \(P(X=x)\) zu. Diese Zuordnung nennt man Wahrscheinlichkeitsmassenfunktion (PMF). Für einen fairen Würfel gilt beispielsweise:
\[
P(X=x) = \frac{1}{6} \quad \text{für } x \in \{1,2,3,4,5,6\},
\]
wobei die Summe aller Wahrscheinlichkeiten
\[
\sum_{x=1}^{6} P(X=x) = 1
\]
sein muss.