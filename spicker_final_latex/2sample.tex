\begin{center}
\fbox{0) Alle Infos 2-Sample Tests:}
MEOOOOOOOOOOOOW

\begin{verbatim}
    /\_____/\
   /  o   o  \
  ( ==  ^  == )
   )         (
  (           )
 ( (  )   (  ) )
(__(__)___(__)__)
\end{verbatim}
\columnbreak
\fbox{I) 2-Sample Gauss Test:}

Hauptziel: Hier wird die Hypothese über die
Mittelwerte ($\mu_1, \mu_2$) getestet
\end{center}

\begin{center}
\fboxyellow{
     {\Large{Means  sind unbekannt, wir kennen $\sigma_1, \sigma_2$}}
}
\end{center}

\large{\textbf{Gegeben muss sein:}}
\[
H_0: \mu_1 = \mu_2, \quad 
H_0: \mu_1 \leq \mu_2, \quad 
H_0: \mu_1 \geq \mu_2
\]

\[
\begin{array}{|c|l|}
\hline
\textbf{Symbol} & \textbf{Bedeutung} \\
\hline
n_1,\,n_2 & \text{Stichprobengrößen} \\
\sigma_1,\,\sigma_2 & \text{SD der gesamtheiten} \\
\overline{X}_{(n_1)},\,\overline{Y}_{(n_2)} & \text{Sample Means} \\
\hline
\end{array}
\]

\normalsize

\large{\textbf{Teststatistik:}}
\[
T = \frac{\overline{X}_{(n_1)} - \overline{Y}_{(n_2)}}
{\sqrt{\frac{\sigma_1^2}{n_1} + \frac{\sigma_2^2}{n_2}}}
\;\;\sim N(0,1)
\]

\large{\textbf{Decision Rule \(R\):}}
\[
T \in R \implies \text{reject } H_0
\]

\large{\textbf{Rejection Region \(R\):}}
\[
\begin{array}{|c|c|}
\hline
H_0 & \text{Rejection Region } R \\ \hline
\mu_1 = \mu_2 
  & (-\infty,\,-u_{1-\tfrac{\alpha}{2}})\cup(u_{1-\tfrac{\alpha}{2}},\,\infty) \\ \hline
\mu_1 \leq \mu_2 
  & (u_{1-\alpha},\,\infty) \\ \hline
\mu_1 \geq \mu_2 
  & (-\infty,\,u_{\alpha}) \\ \hline
\end{array}
\]
\large{\textbf{Beispiel:}}
\begin{rcode}{1} 
m1 <-  c(5.46, 5.34, ..., 5.82)
m2 <- c(5.45, 5.31, 4.11, ..., 4.09)
sd1 <- 0.5
sd2 <- 0.6
n1 <- length(m1)
n2 <- length(m2)
#test the H0: mu1 >= mu2
alpha <- 0.05
#rejection Region
r <- qnorm(alpha)
#[ , -1.644854]
#teststistic
t <- (mean(m1) - mean(m2)) / 
  sqrt((sd1^2 / n1) + (sd2^2 / n2))
#1.027782
p_value <- pnorm(t) 
#0.8479739
#we fail to reject H0 since we are outside of the rejection area
\end{rcode}

\textbf{P-value Berechnen:}
\begin{rcode}{1}
pnorm(t) #H0: mu1 >= mu2
1-pnorm(t) #H0: mu1 <= mu2
2*pnorm(-abs(t)) #H0: mu1 = mu2
\end{rcode}

\columnbreak
\normalsize
\begin{center}
\fbox{II) 2-Sample t-Test (Varianzen gleich und unbekannt):}
\normalsize
Hauptziel: Hier wird die Hypothese über die
Mittelwerte ($\mu_1, \mu_2$) getestet
\end{center}
\normalsize
\begin{center}
\fboxyellow{
     {\Large{Means $\mu_1, \mu_2$ sind unbekannt und $\sigma_1 = \sigma_2$}}
}
\end{center}

\large{\textbf{Gegeben muss sein:}}
\[
H_0: \mu_1 = \mu_2, \quad 
H_0: \mu_1 \leq \mu_2, \quad 
H_0: \mu_1 \geq \mu_2
\]
\textcolor{red}{\warning} Es muss für x und y ein Sample gegeben sein \textcolor{red}{\warning}
\large{\textbf{Beispiel:}}
\begin{comment}
\[
\begin{array}{|c|l|}
\hline
\textbf{Symbol} & \textbf{Bedeutung} \\
\hline
n_1,\,n_2 & \text{Stichprobengrößen} \\
\overline{X}_{(n_1)},\,\overline{Y}_{(n_2)} & \text{Sample Means} \\
S^2_{X,n_1},\, S^2_{Y,n_2} & \text{Sample SDs} \\
\hline
\end{array}
\]

\normalsize

\large{\textbf{Pooled Sample Variance:}}
\[
S_p^2 \;=\; 
\dfrac{(n_1 - 1)\,S_{X,n_1}^2 + (n_2 - 1)\,S_{Y,n_2}^2}
{n_1 + n_2 - 2}
\]

\large{\textbf{Teststatistik:}}
\[
T = \dfrac{\overline{X}_{(n_1)} - \overline{Y}_{(n_2)} \;-\; (\mu_1 - \mu_2)}
{S_p\,\sqrt{\dfrac{n_1 + n_2}{n_1\,n_2}}}
\;\;\sim t_{\,n_1 + n_2 - 2}
\]

\large{\textbf{Decision Rule \(R\):}}
\[
T \in R \implies \text{reject } H_0
\]

\large{\textbf{Rejection Region \(R\):}}
\[
\begin{array}{|c|c|}
\hline
H_0 & \text{Rejection Region } R \\ \hline
\mu_1 = \mu_2 
  & (-\infty,\,-t_{n_1+n_2-2,\,1-\tfrac{\alpha}{2}}) \cup (t_{n_1+n_2-2,\,1-\tfrac{\alpha}{2}},\,\infty) \\ \hline
\mu_1 \leq \mu_2 
  & (t_{n_1+n_2-2,\,1-\alpha},\,\infty) \\ \hline
\mu_1 \geq \mu_2 
  & (-\infty,\,-t_{n_1+n_2-2,\,1-\alpha}) \\ \hline
\end{array}
\]
\end{comment}
\begin{rcode}{1}
x <- c(7.06, 11.84, ..., 8.54)
y <- c(8.68, 6, 7.82, 4.7, ..., 12.36)
#H0: X >= Y, H1 x < y
###################################
#case: equal variances:
t.test(x, y, alternative = 'less', paired = F, var.equal = T, conf.level = 0.95)
#p-value = 0.0181
#we can reject the h0 since we are under 0.05
\end{rcode}
\normalsize
\begin{center}
\fbox{III) Welch test (Varianzen ungleich, aber unbekannt):}
\end{center}
\normalsize
Hauptziel: Hier wird die Hypothese über die
Mittelwerte ($\mu_1, \mu_2$) getestet
\normalsize
\begin{center}
\fboxyellow{
     {\Large{Means \textcolor{blue}{$\mu_1, \mu_2$} sind unbekannt und $\sigma_1 \neq \sigma_2$}}
}
\end{center}

\large{\textbf{Gegeben muss sein:}}
\[
H_0: \mu_1 = \mu_2, \quad 
H_0: \mu_1 \leq \mu_2, \quad 
H_0: \mu_1 \geq \mu_2
\]
\textcolor{red}{\warning} Es muss für x und y ein Sample gegeben sein \textcolor{red}{\warning}
\large{\textbf{Beispiel:}}
\begin{rcode}{1}
x <- c(7.06, 11.84, ..., 8.54)
y <- c(8.68, 6, 7.82, 4.7, ..., 12.36)
#H0: X >= Y, H1 x < y
###################################
#case: unequal variances:
t.test(x, y, alternative = 'less', paired = F, var.equal = F, conf.level = 0.95)
#p-value = 0.01596
#we can reject the h0 since we are under 0.05
\end{rcode}
\fboxyellow{\textcolor{red}{\warning} Das einzige was sich ändert ist: var.equal = F\textcolor{red}{\warning}}

\columnbreak
\begin{center}
    
\fbox{IV) Two Paired Sample t-Test}
\end{center}
\textcolor{red}{\warning}Wenn Z.B einzelne Partienten vorher nacher\textcolor{red}{\warning}

\normalsize
Hauptziel: Wir berechnen als erstes den Unterschied aller Werte der beiden Samples, und dann schauen ob der Mean signifikant Unterschiedlich von 0 ist.
\normalsize
\begin{center}
\fboxyellow{
     {\Large{$\sigma$ ist unbekannt}}
}
\end{center}

\large{\textbf{Gegeben muss sein:}}
\[
H_0: \mu_1 = 0, \quad 
H_0: \mu_1 \leq 0, \quad 
H_0: \mu_1 \geq 0
\]
\textcolor{red}{\warning} Es muss für x und y ein Sample gegeben sein \textcolor{red}{\warning}
\large{\textbf{Beispiel:}}

-
\begin{rcode}{1}
#H0: mu = 0, H1: mu != 0
x <- c(16, 15, 11, 20, ..., 15, 14, 16)
y <- c(13, 13, 10,.., 10, 15, 11, 16)
t.test(x = x, y = y, alternative = 'two.sided', paired = T, var.equal = T, conf.level = 0.95, mu = 0)
#0.0007205
\end{rcode}
\fboxyellow{\textcolor{red}{\warning} Das einzige was sich ändert ist: paired = T\textcolor{red}{\warning}}

\begin{center}
    
\fbox{V) Testing two Variances - F Test}
\end{center}
\normalsize
Hauptziel: Wir vergleichen die beiden sample Varianzen.
\normalsize
\begin{center}
\fboxyellow{
     {\Large{$\sigma$ ist unbekannt}}
}
\end{center}

\large{\textbf{Gegeben muss sein:}}
\[
H_0: \sigma_1 = \sigma_2, \quad 
H_0: \sigma_1 \leq \sigma_2, \quad 
H_0: \sigma_1 \geq \sigma_2
\]
\textcolor{red}{\warning} Es muss für x und y ein Sample gegeben sein \textcolor{red}{\warning}
\large{\textbf{Beispiel:}}
\begin{rcode}{1}
x <- c(102.4, 101.3, ..., 100.1)
y <- c(98.4, 101.7, ..., 101.0)
#H0: sd_x <= sd_y, H1: sd_x > sd_y
alpha <- 0.05
var.test(x = x, y = y, alternative = 'greater', conf.level = 1-alpha)
#p-value = 0.03404
\end{rcode}

\large{\textbf{Beispiel:}}
\normalsize
Erst H0 dass vars gleich sind. Wenn nicht reject, dann müssten wir mein Mean test, var.equal auf True
\begin{rcode}{1}
#H0: sigma_x = sigma_y, H1: sigma_x != sigma_y
var.test(x = x, y = y, alternative = 'two.sided', conf.level = 1-alpha)
#p-value = 0.8814
#we fail to reject H0, also Varianz gleich oder fast gleich
###############################################
#H0: mu_x <= mu_y, H1: mu_x > mu_y
alpha <- 0.025
t.test(x = x, y = y, alternative = 'greater', paired = F, var.equal = TRUE, conf.level = 1-alpha)
#p-value = 0.02374
#we reject H0
\end{rcode}
