\section{\fbox{Probability - Beispiele}}
\subsection{\fbox{Beispiel 1}}
A biased coin (\textcolor{red}{head with probability 1/3})head with probability 1/3) is tossed. \textcolor{red}{If the coins shows tail a fair die is rolled 5 times}
and if the coin shows head \textcolor{red}{a biased die (6 with probability 0.4) is rolled 5 times}. The number of sixes
are counted.
\subsubsection{a) Determine the density of the random X which counts the number of sixes.}
\begin{rcode}{1}
result <- tibble(
    num = 0:5,
    head = dbinom(num, 5, prob = 0.4),
    tail = dbinom(num, 5, prob = 1/6),
    dens = head * 1/3 + tail * 2/3
)
result
\end{rcode}
\subsubsection{b) Evaluate the expected value and the variance of the random variable X.}
\begin{rcode}{1}
expected <- sum(result$num * result$dens)
expectedX <- sum(result$num^2 * result$dens)
var <- expectedX - expected^2
var
\end{rcode}
\subsubsection{c) What is the probability that the coin had shown a head if 3 sixes has been in the 5 rolls?}
$$
P(\text{Head}\mid \text{3 Sixes}) = \frac{P(\text{Head} \land\text{3 Sixes})}{P(\text{3 Sixes})}
$$
\begin{rcode}{1}
p_head_and_three <- 1/3 * dbinom(3, 5, prob = 0.4)
p_three <- 0.098233471 #dens wert für 3 aus der a
p_head_when_threes <- p_head_and_three / p_three
\end{rcode}
\textbf{Hier als oneliner:}
\begin{rcode}{1}
(1/3 * dbinom(3, 5, prob = 0.4)) / 0.098233471
\end{rcode}
\columnbreak
\subsection{\fbox{Beispiel 2}}
In a particular town 10\% of the families have no children, 20\% have
one child, 50\% have two children, 20\% have 3 children and 10\% have
4 children. \textcolor{purple}{Let T represent the total number of children}, \textcolor{olive}{and G the
number of girls}, in a family chosen at random from this town
\subsubsection{a}
Assuming that children are equally like to be boys or girls find
distribution of G.
\begin{rcode}{1}
tibble(
  g = 0:4) %>% rowwise() %>% 
  mutate(
    dens = sum(c(0.1, 0.2, 0.4, 0.2, 0.1) * 
                 dbinom(g, size = 0:4,
                 prob = 0.5))))
      g    dens
  <int>   <dbl>
1     0 0.331  
2     1 0.4    
3     2 0.213  
4     3 0.05   
5     4 0.00625
\end{rcode}
\textbf{g = 0:4 :} gibt an wieviele Kinder die Familie hat.\\
\textbf{dens :} Für jede Berechung von dens ist g eine Statische Zahl. Z.b 2.\\
Wir berechnen die Prob das eine Familie 0-4 Kinder hat MAL die Prob das eine Familie mit 0-4 Kindern 2 Mädchen hat. Das wäre 0.213
\subsubsection{b}
f you know that 2 girls are in an arbitrary chosen family, find the
probability that the family has 4 children\\
\textbf{A:}
Hier brauchen wir Bayes theorem
$$
P(\text{4 Children}\mid \text{2 Girls}) = \frac{P(\text{4 Children} \land\text{2 Girls})}{P(\text{2 Girls})}
$$
\begin{rcode}{1}
# P( 4 children | 2 girls) = P(4 child und 2 gils) / P( 2 girls)
p_4child_2girls <- 0.1 * dbinom(2, 4 , 0.5)
p_2girls <- 0.213
p_4child_when_2girls <- p_4child_2girls / p_2girls
p_4child_when_2girls
\end{rcode}
\subsubsection{b}
Suppose the names of all children in the town are put into a hat,
and a name is picked out at random. Let U be the total number
of children in the family of the child picked at at random. Find
the distribution of U.
\begin{rcode}{1}
(0:4) * c(0.1, 0.2, 0.4, 0.2, 0.1) / sum((0:4) * c(0.1, 0.2, 0.4, 0.2, 0.1))
#0.0 0.1 0.4 0.3 0.2
\end{rcode}
jedes $N$ an Kindern die eine Familie haben kann Teilen wir durch das Expected Value an Kindern. So kennen wir die Chance das die Familie die wir ziehen $N$ Kinder hat