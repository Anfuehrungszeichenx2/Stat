\begin{multicols*}{2}

\begin{center}
     \Large{Confidence Interval for $\hat{p}$ (Proportion)(Stichprobenanteil)}
\end{center}

\begin{center}
     \Large{We are estimating $\hat{p}$, the sample proportion}
\end{center}

\[
\left[ \hat{p} - u_{1 - \frac{\alpha}{2}} \cdot \sqrt{\frac{\hat{p}(1 - \hat{p})}{n}}, \, \hat{p} + u_{1 - \frac{\alpha}{2}} \cdot \sqrt{\frac{\hat{p}(1 - \hat{p})}{n}} \right]
\]
$$\hat{p} = \frac{probability \cdot n}{n}$$

\begin{lstlisting}
prob <- 0.7
alpha <- 0.05
n <- 250
p_hut <- (prob*n)/n
q <- qnorm(1-(alpha/2))
L <- p_hut - q * sqrt((p_hut * (1-p_hut))/n)
U <- p_hut + q * sqrt((p_hut * (1-p_hut))/n)
binom.test(x=0.7 * n, n=n, conf.level = 1-alpha, alternative = "two.sided")$conf.int
#[0.6431948, 0.7568052]
\end{lstlisting}
wir sind uns zu 95\% sicher, dass zwischen 64\% und 75\% der Wähler ja gestimmt haben.
\bigbreak

\fbox{Nur obere oder untere Grenze berechnen:}

Hier teilen wir $\alpha$ nicht mehr, da sich die Prozente auf eine Seite konzentrieren:

\begin{lstlisting}
q <- qnorm(1 - alpha)
L_alleine <- p_hut - q * sqrt((p_hut * (1 - p_hut))/n)
L_exact <- binom.test(x=0.7*n, n=n, conf.level = 1-alpha, alternative = 'greater')$conf.int
U_alleine <- p_hut + q * sqrt((p_hut * (1 - p_hut))/n)
U_exact <- binom.test(x=0.7*n, n=n, conf.level = 1-alpha, alternative = 'less')$conf.int

\end{lstlisting}

\hrule

\begin{center}
     \Large{Umformungen}
\end{center}

\fbox{$\hat{p}$: Solving for $\hat{p}$}
\[
\hat{p} = \frac{\text{upper limit} + \text{lower limit}}{2}
\]

\begin{lstlisting}
p_hut <- (L + U) / 2
\end{lstlisting}
\bigbreak
\fbox{$u_{1 - \frac{\alpha}{2}}$: \small{Für Quantile der Normalverteilung Umformen}}

Aus der Intervalll\"ange:
\[
u_{1 - \frac{\alpha}{2}} = \frac{\text{Intervalll\"ange} \cdot \sqrt{n}}{2 \cdot \sqrt{\hat{p}(1-\hat{p})}}
\]


Aus der oberen Grenze:
\[
u_{1 - \frac{\alpha}{2}} = \frac{\text{obere Grenze} - \hat{p}}{\sqrt{\frac{\hat{p}(1-\hat{p})}{n}}}
\]

Aus der unteren Grenze:
\[
u_{1 - \frac{\alpha}{2}} = \frac{\hat{p} - \text{untere Grenze}}{\sqrt{\frac{\hat{p}(1-\hat{p})}{n}}}
\]

\begin{lstlisting}
#nach q umstellen
leange <- U - L
q_umgestellt_1 <- (leange * sqrt(n)) / (2 * sqrt(p_hut * (1 - p_hut)))
q_umgestellt_2 <- (U - p_hut) / (sqrt(p_hut * (1 - p_hut) / n))
q_umgestellt_3 <- (p_hut - L) / (sqrt(p_hut * (1 - p_hut) / n))
\end{lstlisting}



\fbox{MOE (Margin of Error):}
\[
MOE = u_{1 - \frac{\alpha}{2}} \cdot \sqrt{\frac{\hat{p}(1-\hat{p})}{n}}
\]
$$
MOE = \frac{Intervallänge}{2}
$$

\begin{lstlisting}
moe <- q * (sqrt(p_hut * (1 - p_hut) / n))
moe_2 <- leange / 2
\end{lstlisting}

\fbox{Intervalll\"ange:}
\[
\text{Intervalll\"ange} = 2 \cdot u_{1 - \frac{\alpha}{2}} \cdot \sqrt{\frac{\hat{p}(1-\hat{p})}{n}}
\]
$$
\text{Intervalll\"ange} = U - L
$$

\begin{lstlisting}
intervalllaenge <- 2 * q * (sqrt(p_hut * (1 - p_hut) / n))
intervalllaenge <- U - L
\end{lstlisting}

\end{multicols*}